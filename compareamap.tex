\documentclass[12pt,letterpaper]{article}
\usepackage{setspace}
\singlespace
%\onehalfspace
%\doublespace
%\setlength{\parindent}{1cm}
\usepackage[utf8]{inputenc}
\usepackage{amsmath}
\usepackage{amsfonts}
\usepackage{amssymb}
\usepackage{hyperref}
\usepackage{color}
\definecolor{titlegreen} {RGB}{110,145,60}
\hypersetup{
	%driver=pdftex,
	colorlinks=true,
	urlcolor=blue,
	linkcolor=blue,          % color of internal links
    citecolor=blue,        % color of links to bibliography
    filecolor=magenta
}
\usepackage[sorting=nyt,natbib=true,citestyle=authoryear,bibstyle=authoryear,maxnames=3]{biblatex}
\bibliography{proposal}
\defbibheading{bibliography}{%
	\section{References}
	}

\title{Compare-a-Map}
\author{Alex Mandel}
	
\begin{document}
\maketitle
\tableofcontents

\section{Introduction}
The majority of web-based interactive maps use one of the widely-available 3rd party cartographic products (e.g. Google, Mapquest, OpenStreetMap, etc.) as the base layer (more commonly referred to as a base map) for which to construct a map. Deciding which base map to use involves checking to ensure a chosen option adequately displays the features of interest and is visually complimentary with custom cartographic overlays.  Presented here is a multiple map comparison tool allowing web cartographers to compare side-by-side the majority of available maps along with more customizable options on a single web page. Panning and zooming is linked between the maps to ensure the accuracy of the comparison. Users can focus on comparing cartographic style and data availability as the multiple map tool handles the differences in application programming interfaces (API), projections, and geocoding.  Providing an easy comparison via the small multiple methodology should assist in the choosing \parencite{Tufte1990} of the most appropriate mix of base layers for the construction of a web map.
%Maps/Map oriented interfaces are becoming more common in electronic form (cite) but the field of Cartography as historically practiced is considered by some to be dead (Wood?). Many people are getting into the business/recreation of making maps and most of this is online.

	To understand why such a comparison is necessary one has to take into account that the creation of interactive cartographic products for the web requires a complex set of software collectively referred to as Spatial Data Infrastructure (SDI). Realizing just how much time it takes to learn and configure such an infrastructure makes it clear that it's much easier when a basic map is already provided for you; there is no need to learn projections, find data, or run a web server. However just because the foundation layers of the map are provided does not mean they are best suited to a given project and too often popularity is the deciding factor at a cost to quality. There are choices, Google is not the only map provider as visitors of many websites will note \footnote{A random selection of Store Location tools on various business websites demonstrates this point fairly well} there is a wide variety of tools in use. Generally the non-profit or casual user can freely move between systems but for businesses and organizations wanting to include maps this decision will impact their budgets through  licensing fees and IT staff time requirements. In the world of mash-ups this appears to result in a significant use of Google Maps, the first to release a free public Application Programming Interface(API) \parencite{Turner2006}, amongst neographers while many businesses have continued to use other products they had a previous relationship with or possibly due to better licensing terms.
	  
	  The Compare-A-Map tool intends to extend the discourse on cartography in the digital age along the trend set by such tools as ColorBrewer \parencite{Harrower2003}, Typebrewer \parencite{Sheesley2008}, Mapshaper \parencite{Bloch2006}, FlexProjector \parencite{Jenny2010} , etc. It is unique in that it's the first instance I know of to compare more than 2 map providers on the same page at the same time and intends to cover all of the major providers. As an added bonus the website also demonstrates how to create a webmap with any of the data sources as all the code; Javascript, HTML, CSS, is avialable under a free and open source license directly from the site.
	
%Like Colorbrewer \parencite{Brewer2003}, Typebrewer\parencite{Sheeshley2008}, Cartography2.0, Mapshaper, etc. a tool to help cartographers (anyone who makes a map whether they have training in cartography). Allow a user to simultaneously compare the "base" maps available through the major API providers;  Google, Bing, Yahoo, Mapquest, OpenStreetMap.


\section{Methods}
Compare-A-Map (\url{http://maps.ice.ucdavis.edu/compare-a-map}) is a website. A single webpage that shows 5 map providers side by side;
% fix
Google, Bing, Yahoo, Mapquest(OpenStreetMap Data), and OpenStreetMap. It is built using Free and Open Source tools available to any web developer; Openlayers, HTML, CSS, JQuery and Apache HTTP.

There are several ways to use such a tool, to see what data exists for your neighborhood, explore some interesting place, or to find the right starting point for you to use as a canvas on which to construct your own map. This paper focuses on using Compare-A-Map to evaluate and choose an appropriate basemap to use in a webmap you want to construct.

\section{How to use}
All of the choices you'll make in the interface are based on the goal of the map you want to make.

\subsection{Basic Usage}
\begin{enumerate}
\item Pick what map type you want to compare - Street, Satellite (aka AirPhoto), Terrain or OpenStreetMap.
\item Optionally type in a place to recenter the map - City, State (Full addresses do not always work with the Nominatim backend) %Cite Nominatim
\item Optionally change the zoom level with the slider to show more detail or a wider area.
\item Pan and zoom the maps as necessary to review the comparison you are interested in.
\end{enumerate}

\subsection{Basic Questions}
These are generic questions to ask yourself when critiquing each basemap for it's suitability in the end map you envision creating.
\item Is there data for the area of interest?
\item Is the data from an appropriate time period?
\item Are points of interest and landmarks your end users would recognize in the map and easily discrenable?
\item Will the basemap conflict with your overlay data you plan to add? This could be because there is too much information on the page, or the color palette is too similar, too different, or doesn't leave you any options for colors not already in use.

\subsection{What to Look For}
\subsubsection{Street}
Street view is the most common default on all the mapping services. It's often used as the base for navigation and reference to the built environment. The most obvious difference will be the color selection, typography and type placement. More subtle differences include data availability (which includes selective data inclusion and lack of data), data quality, place and street name differences.  

\subsubsection{Satellite}
This page compares aerial photography, while not always from Satellites that is what most of the map providers label it as. The key features to pay attention to on this page is the colors, seasonality, year, maximum resolution (how many inches/meters per pixel when zoomed in as close as possible).

\subsubsection{Terrain}
The definition of what makes a terrain map seems to vary the widest. Many of the providers don't provide a separate terrain map but put it into mid level regional zooms of their Street layer. The most commonly shared feature is hill-shading effects, and color tones that reflect nature over the urban environment. Things to compare include color scheme, presence/absence of contour lines, and labeling of National Forest, National Parks, and other significant natural resources.

\subsubsection{OpenStreetMap}
This genre highlights what is possible when you build your own basemap with available data. In this case OpenStreetMap data is licensed in such a way that it is free to everyone. The only requirement of usage is proper attribution to the data source. Interesting features of OpenStreetMap include the focus on footpaths, cycle paths, trails and other non auto oriented features. It also boasts really good data in unexpected places where commercial providers haven't had enough financial incentive to create or provide data.




\subsection{Additional Features:}
\begin{itemize}
	\item The site is designed to fit on a single screen for most desktop, laptop, and netbook computer users in the world (1000x600 or less pixels).
	\item If you pan any of the windows the others move to match. \footnote{There is a known bug in the Yahoo code right now so moving that window does not work correctly}
	\item If you zoom in or out using the scroll wheel or the scroll toggle provided on screen all of the maps zoom together. It only goes between the zooms that all the providers share 1-15 (Google and some of the others go to 18 - finest detail). Zoom levels are approximately similar resolutions/map scale but are not identical between the providers.
	\item The maps can be re-centered using a place-name geocoder search box, which currently is only tested with City names and zip codes. Nominatim 	\parencite{Mapquest2012} was picked as the geocoding tool because it's a unique data source with much more liberal licensing based on OpenStreetMap data, the downside is that it does not fully support address level geocoding yet. 
\end{itemize}


\subparagraph{Additional Planned Features:}
\begin{itemize}
	\item Provide a short tutorial for users on the most effective way to use the tool.
	\item Provide some example comparisons that demonstrate that even though the data is similar between the sources it's not the same and the fundamental cartography is different. This includes but is not limited to fonts, colors, scale dependent labels, inclusion and exclusions of non-road features.
	\item A similar one page comparison of aerial imagery and another of hybrid (aerial with road data over-layed).
	\item Provide a suggestion submission mechanism to take feedback and ideas from visitors.
	\item While the code running the site is currently accessible to people who know how websites work, it will be better to publish the code on a code sharing website for easier download.
	\item Clicking on the Title of a mini-map will take you to the same center point on the service provider's full screen home map.
\end{itemize}

%This doesn't cover geocoding but could, although several studies have been completed on that topic (cite) as the precision is much easier to compare numerically than the aesthetics of the cartography. A sub component which has been studied somewhat also is the accuracy of the road network data, but that too is highly variable depending on the service and where in the world you compare. Hence this tool does provide a means of also visually comparing.


\section{Conclusion}
\subsection{Future}
\begin{itemize}
	\item Geocode compare (ref ucla or was it usc work)
	\item License compare
	\item to see more visit the project code repository on github
\end{itemize}

\printbibliography[heading=bibliography]

\end{document}