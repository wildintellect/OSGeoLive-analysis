\documentclass[12pt,letterpaper,draft]{article}
\usepackage[latin1]{inputenc}
\usepackage{amsmath}
\usepackage{amsfonts}
\usepackage{amssymb}
\author{Alex Mandel}
\title{Bridging the geospatial technological divide with free and open source software"}
\begin{document}
\section{OSGeo Live}
Making it easier for anyone to try geospatial tools by eliminating the need to install.
Live DVD, Live USB, Live through the web
Available in multiple languages with modest hardware requirements
Analyze the download data to better understand who it's reaching (Also get similar data from OSGeo attendance)
Survey (Opt-in) of users of the disc \& web visitors
What role they play in their 'company' implementer, decision maker, researcher, grunt
Prior experience with GIS
Prior experience with Open Source
Prior experience with Open Source GIS
Expectations

Background: Adoption of Open Source solutions in India, Spain, Brazil etc.. Skipping the prior method of buying into the Commercial Product.
Note: Funding of the development of the solutions early on to ensure they meet the needs.

\section{Everday Cartography Tools}

\subsection{Projection Assistant}
A tool to help someone select an appropriate projection for a map they want to make. Based on the USGS chart of distortions by map projection and including new projections like Spherical Mercator walk the user through a process of selecting a projection based on where they are mapping, how big of an area they are mapping and what distortion they want to minimize.
Web based app and qgis plugin.


\subsection{Open Symbology}
A new tool to embed keywords in iconography and for searching and cataloging place markers by set, theme, content in an open manner that can be used by anyone.
Focuses on the conversion freely available icons to more usable formats.
Example: National Park service Ai/pdf/font to svg
Web based app and qgis plugin.

\section{Engaging the Student via Community Contributed Mapping}
Openstreetmap Curriculum for higher ed
	Developing a 1-2 lab session lesson on the role of Volunteered Geographic Information(VGI, aka User contributed data) in the realm of GIS and geo-awareness.

	Layout:
		View OSM
		Download OSM
		Analyze or Map with OSM
			Style an OSM map
			Create a web map with OSM
			Printed OSM data (Walking Papers)
		Edit OSM
		Push changes back to OSM

	How OpenStreetmap(Indirectly) pushed Google to allow for submission of corrections and POIs.

	How modern tools, and everyday web, phone applications are based in spatial cognition.
	Building on that to provide geography education
	Crowd sourcing, Virtual Globes, Location awareness

	Test this lesson plan in the classroom. Possible before and after questionarre to gauge what students think about VGI, it's importance in the future of GIS and the issues that arise (governence, quality, etc.)


\section{Web Mapping for the community group}
	Building community website around a geospatial component for and by communities. What do communities really need, how to provide such a tool, test for ease of use.
	Web based is the future...
		Increase in always connected low power devices - netbooks, tablets, smartphones.
		Trend: Google Maps, GPS Navigation,  
	Potential for Citizen Science, cultural preservation, eco/agro/etc/tourism.
	Making the case for how Open Source makes this easier, either by access to tools or by forcing the competition to offer better products. Similar to Google's disruptive technology methodology, offering something better for free.

\end{document}