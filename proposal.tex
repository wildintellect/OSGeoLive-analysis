\documentclass[12pt,letterpaper]{article}
\usepackage{setspace}
\singlespace
%\onehalfspace
%\doublespace
%\setlength{\parindent}{1cm}
\usepackage[utf8]{inputenc}
\usepackage{amsmath}
\usepackage{amsfonts}
\usepackage{amssymb}
\usepackage{hyperref}
\usepackage{color}
\definecolor{titlegreen} {RGB}{110,145,60}
\hypersetup{
	%driver=pdftex,
	colorlinks=true,
	urlcolor=blue,
	linkcolor=red,          % color of internal links
    citecolor=titlegreen,        % color of links to bibliography
    filecolor=magenta
}
\usepackage[sorting=nyt,natbib=true,citestyle=authoryear,bibstyle=authoryear,maxnames=3]{biblatex}
\bibliography{proposal}
\defbibheading{bibliography}{%
	\section{References}}

\author{Alex Mandel}
\title{Bridging the geospatial technological divide with free and open source software
\\	
\textsc{\small Preliminary Research Proposal}\\[0.5cm]
}

\begin{document}
\maketitle
\section{Abstract}

	With the advent of the internet, gps enabled devices and... there has been an explosion in Location based services, Mashups,Spinning Globes, Geospatial Science, Geographic Information Systems(GIS), etc . The power of quality map making is no longer limited to the 'State' but is now in the hands of anyone who has a story to tell or agenda to push \parencite{Wood2010}. The democratization of geospatial knowledge like many other knowledges in the Internet age has great potential to positively influence the lives of many citizens. However, just because the tools exists, many of them free of monetary cost, does not mean that everyone can overcome the "...financial, temporal, and experience and skills barriers that can impede access to and use of GIS and digital data..." \parencite{Elwood2006}. 

%"Researchers have identified the financial, temporal, and experience and skills barriers that can impede access to and use of GIS and digital data by resource poor organizations and social groups (Weiner et al. 1995, Sawicki and Craig 1996, Barndt 1998, Craig and Elwood 1998)." .

	There are several approaches to exploring the topic of making map related technologies more acccessible to more people. The approach depends partly on the GIS and Information Technology experience and the profession, if map making is their job, of the end user. The following studies aim to provide an easy entrance for assessing the capabilities and learning of Geospatial software, tools to assist in the creation of better web based maps, and better defining the specific technical barriers that impede communities from building collaborative maps on their own. Particular interest is focused on Free and Open Source Software with the notion that the licensing and community structure around such products in itself reduces some financial, temporal, and skill barriers while also sharing the philosphical principles ??? 

\tableofcontents
\section{OSGeo Live}

\subsection{Introduction}
	Depending on who you ask in the field of GIS historically (pre 2000) there have only been a handful of software options primarly targeted at the personal computer desktop user. With the growth of the Internet and information technology there's been a growth in software options for the desktop but also the web and mobile platforms. A significant part of this growth has been in Free and Open Source software.
	A substantial barrier to software adoption in general, especially in regards to web services is the technical knowledge and time to install. In particular the time to learn how to setup and configure different applications prevents users from comparing all of their options before committing to a product. 
	From an academic perspective solving this issue would also open up possibilities for expanding instruction in geospatial technologies. Being able to provide a variety of software options ready for use in computer laboratories and for student use at home...

This project takes advantage of the ability to freely distribute Free and Open Source Software to ensure that anyone can download and use the software.

%The initial goal of OSGeo Live was to make it easier for anyone to try geospatial tools, particularly Free and Open Source ones, by eliminating the need to install.
%I've been involved for over a year in developing a GIS workstation available in a variety of formats:
%a Live DVD, Live USB, Virtual Machines and live through the web. The workstation consists entirely of Free and Open Source Software (Over 40 applications), is available in multiple languages, contains sample data and only requires modest hardware to run. We've released 3 major revisions and are on schedule to create a new release every 6 months. Over 1000 copies have been handed out physically at the annual FOSS4G conference and just as many have been downloaded.
%
%There are lot of possible uses of OSGeo Live and in particular I want to focus on it's usefulness as a teaching tool. I've already piloted it's usage for a 2 hour lab as part of the International Seminar on Climate Change and Natural Resources Management through the US Forest Service and facilitated by ICE. It's also becoming the norm for workshops at FOSS4G. However I have yet to collect any formal feedback(informally: interest from developers wanting to add their software and volunteers to help translate), or thoroughly analyze the downloads/conference audience.


\subsection{Methods}
Build the product (Updated every 6 months)
Distribute the product: Internet (DVD ISO image file ready to burn, Virtual Machine compatible with common Virtual Machine software freely available - VMWare player, VirtualBox - , Conferences (As DVD or USB flash drive)
Use in various Education settings:
	FS Workshops
	Classroom Setting
Analyze the distribution/downloading 
Create an online feedback survey - Snowball sampling via website, mailing lists and word of mouth(ex conference booths)
	Better Understand the audience of product
	Who (Fields and Jobs) is using the product Where and for what purpose?
	Are we reaching people who are new to GIS?
	Does localization of material matter (Translation)?	
	
Analyze who has participated in the development of the product. Same survey or interviews?	
	
Methods:
\begin{enumerate}
\item Analyze the download data to better understand who it's reaching (By at least Country level, City/Region where possible) and also obtain similar demographic data from conference attendance records where OSGeo Live is distributed.
\item Survey (Opt-in, could require IRB short form for anonymous participation) of users of the disc \& web visitors
	\begin{enumerate}
		\item What role they play in their 'company': implementer, decision maker, researcher, labor?
		\item Prior experience with GIS
		\item Prior experience with Open Source
		\item Prior experience with Open Source GIS
		\item Expectations
		\item Usage in learning Open Source GIS, GIS in general
		\item Location, Native location if different
		\item Languages, Primary Language if not English
		\item Does providing the documentation and software in a native make it easier to use?
	\end{enumerate}
\end{enumerate}

Additional Background Notes: Adoption of Open Source solutions is occurring in nations like India, Spain, Brazil etc.. but is skipping the prior method of buying into the Commercial Product and jumping directly to use of Free and Open Source software including the funding of customization of existing applications and development of new applications.

\url{http://live.osgeo.org}

%Note: Funding of the development of the solutions early on to ensure they meet the needs.
\section{Compare a Map}
\subsection{Introduction}
The majority of web-based interactive maps use one of the widely-available 3rd party cartographic products (e.g. Google, Mapquest, OpenStreetMap, etc.) as the base layer (more commonly referred to as a base map) for which to construct a map. Deciding which base map to use involves checking to ensure a chosen option adequately displays the features of interest and is visually complimentary with custom cartographic overlays.  Presented here is a multiple map comparison tool allowing web cartographers to compare side-by-side the majority of available maps along with more customizable options on a single web page. Panning and zooming is linked between the maps to ensure the accuracy of the comparison. Users can focus on comparing cartographic style and data availability as the multiple map tool handles the differences in application programming interfaces (API), projections, and geocoding.  Providing an easy comparison should assist in the choosing of the most appropriate mix of base layers for the construction of a web map.
%Maps/Map oriented interfaces are becoming more common in electronic form (cite) but the field of Cartography as historically practiced is considered by some to be dead (Wood?). Many people are getting into the business/recreation of making maps and most of this is online.

	To understand why such a comparison is necessary one has to take into account that the creation of interactive cartographic products for the web requires a complex set of software collectively referred to as Spatial Data Infrastructure. Realizing just how much time it takes to learn and configure such an infrastructure makes it clear that it's much easier when a basic map is already provided for you; there is no need to learn projections, find data, or run a web server. However just because the foundation layers of the map are provided does not mean they are best suited to a given project. Clearly there are choices, Google is not the only map provider as visitors of many websites will note there is a wide variety of tools in use. Generally the non-profit or casual user can freely move between systems but for businesses and organizations wanting to include maps this decision will impact their budgets through  licensing fees and IT staff time requirements. In the world of mashups this appears to result in a lot of use of Google Maps, the first to release a free public Application Programming Interface(API) \parencite{Turner2006}, amongst neographers while many businesses have continued to use other products they had a previous relationship with.
	  This tool extends the discourse on cartography in the digital age along the trend set by such tools as ColorBrewer \parencite{Harrower2003}, Typebrewer \parencite{Sheesley2008}, Mapshaper \parencite{Bloch2006}, FlexProjector \parencite{Jenny2010} , etc. It is unique in that it's the first instance I know of to compare more than 2 map providers on the same page at the same time and intends to cover all of the major providers. As an added bonus the website also demonstrates how to create a webmap with any of the data sources as all the code; Javascript, HTML, CSS, is avialable under a Free and Open Source license directly from the site.
	



%Like Colorbrewer \parencite{Brewer2003}, Typebrewer\parencite{Sheeshley2008}, Cartography2.0, Mapshaper, etc. a tool to help cartographers (anyone who makes a map whether they have training in cartography). Allow a user to simultaneously compare the "base" maps available through the major API providers;  Google, Bing, Yahoo, Mapquest, OpenStreetMap.


\subsection{Methods}
Created a website using Openlayers toolkit (library) which should fit on a single screen for most computer users (dimensions?)
5 Windows are available with:
If you pan any of the windows the other move to match.
There is a place based search tool to recenter the map, currently based on city,st as the reference dataset is not a full geocoder. Nominatim was picked partly because it's a unique data source with much more liberal licensing based on OpenStreetMap data.
A zoom slider is provided, to change the zoom. It only goes between the zooms that all the providers share 1-15 (Google and some of the others go to 18 - finest detail)

\url{http://maps.ice.ucdavis.edu/compare-a-map}

Provide a short tutorial for users on the most effective way to use the tool.

If you're going to make a web map the cartography does matter. Even though the data is similar betweeen the sources it's not the same and the fundamental cartography is different. 
Fonts, colors, scale dependent labels, inclusion and exclusions of non-road features.

I'd like to also do the same for aerial imagery, and hybrid maps. This doesn't cover geocoding but could, although several studies have been completed on that topic (cite) as the precision is much easier to compare numerically than the aesthetics of the cartography. A sub component which has been studied somewhat also is the accuracy of the road network data, but that too is highly variable depending on the service and where in the world you compare. Hence this tool does provide a means of also visually comparing.

%\printbibliography[section=2]
%\section{Everday Cartography Tools}

%\subsection{Projection Assistant}
%A tool to help someone select an appropriate projection for a map they want to make. Based on the USGS chart of distortions by map projection and including new projections like Spherical Mercator walk the user through a process of selecting a projection based on where they are mapping, how big of an area they are mapping and what distortion they want to minimize. \href{http://Spatialreference.org}{Spatialreference.org} is a great start but you can't search based on bounding box or distortion type.
%
%Possibly implemented as a Web based application or QGIS plugin, this project is an evolution from the Indiemapper (\url{http://indiemapper.com} ) online cartography tool, the Flex Projector (\url{http://www.flexprojector.com} ) project and the previously mentioned \href{http://Spatialreference.org}{Spatialreference.org}. The goal is simply to fill a gap in the current projection selection inside of GIS tools much in the same way that ColorBrewer and TypeBrewer have aided color and text selection respectively. 
%
%
%\subsection{Open Symbology}
%A new tool to embed keywords in iconography and for searching and cataloging place markers by set, theme,  and content in an open manner that can be used by anyone. Focuses on the conversion of freely available icons to more usable formats(Public Domain icons from US government agencies- NPS,USGS,USFS, etc). Specifically, this idea relies on the relatively new format of Scalable Vector Graphics (svg) which has actually been a web standardized format for several years. The tool could possibly be implemented as a Web based application or QGIS plugin.
%
%Traditionally symbology in GIS has been distributed as fonts or as raster images (ie jpg, png, gif). Editing of fonts is quite a mystery to most people outside of font graphic artists and can be somewhat ardous to create. Raster images simply don't scale well without loss of quality (pixelation). SVG presents an interesting additional feature on top of it's easier editing, the file itself is a plain text xml implementation. Due to it's adoption for web uses svg good editors exist in both the Commercial and Open Source markets (Adobe Illustrator, Inkscape). Because of this file format, and the svg standard(W3C), embedding of metadata (ie keywords) inside the file itself is possible which should enable the implementation of tools that can semantically catalog the icons based on a search. This search capability should make it much easier to find the icon one wants and to more easily mix and match icons from different sets.
%
%This project would be done in collaboration with the OSGeo Graphics project, a collaboration of artists and GIS software developers to build a library of free GIS iconography.
%\\
%\\ Example: National Park service Ai/pdf/font to svg


%\section{Engaging the Student via Community Contributed Mapping}
%'Neo-geographers' have come up with a lot of interesting contributions to maps in society in the last decade. Some might argue that the contributions of Google Earth, Bing Maps, GPS embedded smartphones etc have had a much more significant impact on the re-interest in mapping amongst businesses and the general public than advancements in traditional GIS.
%
%One such neo-geography project, \href{http://openstreetmap.org}{Openstreetmap.org} has brought in a whole new discussion about user contributed(wiki-style) map creation, termed Volunteered Geographic Information(VGI) by the academic geography community. As of yet there are few implementations of such geographic data collection by 'Official' entities for a variety of reasons concerning data quality, ownership, liability, etc. However in the volunteer/ngo community such data sources have already proven highly valuable (OpenStreetMap usage in Haiti disaster relief).
%
%I think that properly training future GIS professionals in the creation, ethics and usage of such data sets will lead to a better understanding and wider adoption of such data collection models in the future. To facilitate this I propose to write a lesson plan and lab suitable for an introductory GIS or general education geography course at a college level (Community, Tier II or Tier I). The lesson plan will be tested on students at several colleges and a survey will be used to measure and improve the lesson. Most likely the study will include students from CSU Stanislas and UC Davis (Possibly also San Jose State and UCSB).
%
%The course will focus on OpenStreetMap and be able to run entirely from the OSGeo Live DVD/USB.
%
%Outline:
%\begin{enumerate}
%\item View OSM
%\item Download OSM
%\item Analyze or Map with OSM
%	\begin{enumerate}
%	\item Style an OSM map
%	\item Create a web map with OSM
%	\item Printed OSM data (Walking Papers)
%	\end{enumerate}
%\item Edit OSM
%\item Push changes back to OSM
%\end{enumerate}
%Methods: Test this lesson plan in the classroom. Possibly with a before and after questionnaire to gauge what students think about VGI, it's importance in the future of GIS and the issues that arise (governance, quality, ethics, etc.)
%
%Additional Background Notes:
%\\ How OpenStreetmap(Indirectly) pushed Google to allow for submission of corrections and POIs.
%\\ How modern tools, and everyday web, phone applications are based in spatial cognition.
%\\ Building on the everyday relevance to inspire geography education.
%\\ Keywords: Crowd sourcing, Virtual Globes, Location awareness

\section{Web Mapping for the community group}
\subsection{Introduction}
	



	There are a lot of reasons why a group of people might want to build a collaborative web map. While there are many paid and free tools available they have restrictions to their uses either by technological barriers, by licensing barriers, or cost barriers. Additionally I want to ensure that the options available to communities make it easy to collaborate with researchers when wanted. That primarily entails ensuring that the backend data storage and export options are of an acceptable quality and format for GIS analysis (e.g.: KML from Google does not satisfy this requirement adequately).
	
\parencite{Dunn2007} - use of Free and Open Source is recommended if the writing of new GIS algorithms is necessary to implement more participatory research agendas because it provides the underlying software components to create a usable GIS system but has no restrictions on it's adaptation.

\subsection{Methods}
My goal is to help identify the barriers to building a community website around a geospatial component for and by communities. The primary focus of the direct user surveys will be to asses aspects of website design both visually (cartographic) and contextually (general web site usability issues - ie can you find what you are looking for)
\begin{enumerate}
\item What do communities really need?
\item How to provide such a too?
\item Methods for testing ease of use?
\end{enumerate}

Possible Methods(Most will require IRB approval):
\begin{enumerate}
\item Community Meetings
%\item Focus Groups (Could be the same as above)
\item Interviews
\item Surveys
	\begin{enumerate}
	\item Initial Survey of Goals and Vision for the Website
	\item Survey of "Editors" experience using the website to enter their data and for finding information
	\item Survey of "Users" experience using the website to find information
	\end{enumerate}
\item Usability Study (Recording of subjects using the interface) - For each user invited to the testing their session will be tracked as they use the website to see the patterns used in interacting with site, how long they spend on given pages,etc.
	\begin{enumerate}
	\item Give the "Editors" a task to complete, the survey will follow the test
	\item Give the "Users" a task to complete, the survey will follow the test
	%Give them the same task to search for things? Plan a trip? Then compare?
	\end{enumerate}
\end{enumerate}


Why the demand for web based map tools only going to increase in the near future:
\begin{itemize}
\item Increase in always connected low power devices - netbooks, tablets, smartphones.
\item Trend: Google Maps, Google Earth, GPS Navigation, IPhone and Android Apps  
\end{itemize}
Possible Applications derived from Community Groups:
\begin{itemize}
\item Potential for Citizen Science
\item cultural preservation
\item eco/agro/etc/tourism
\end{itemize} 

I'm currently in discussion with UC Extension Specialists and the Placer Grown organization in Placer County to develop a participant driven agro-tourism web map. The community process to create this map could serve as the basis of answering the proposed question in a participatory research manner.

Additional Background Notes:
Making the case for how Open Source makes this easier, either by access to tools or by forcing the competition to offer better products. Similar to Google's disruptive technology methodology, offering something better for free.

\printbibliography[heading=bibliography]

\end{document}