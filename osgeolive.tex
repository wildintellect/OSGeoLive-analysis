\documentclass[12pt,letterpaper]{article}
\usepackage{setspace}
\singlespace
%\onehalfspace
%\doublespace
%\setlength{\parindent}{1cm}
\usepackage[utf8]{inputenc}
\usepackage{amsmath}
\usepackage{amsfonts}
\usepackage{amssymb}
\usepackage{hyperref}
\usepackage{color}
\definecolor{titlegreen} {RGB}{110,145,60}
\hypersetup{
	%driver=pdftex,
	colorlinks=true,
	urlcolor=blue,
	linkcolor=blue,          % color of internal links
    citecolor=blue,        % color of links to bibliography
    filecolor=magenta
}
\usepackage[sorting=nyt,natbib=true,citestyle=authoryear,bibstyle=authoryear,maxnames=3]{biblatex}
\bibliography{proposal}
\defbibheading{bibliography}{%
	\section{References}
	}

\title{OSGeo Live}
\author{Alex Mandel}
	
\begin{document}
\maketitle
\tableofcontents

\section{Introduction}
	Historically (pre 2000) there have only been a handful of geospatial software options primarily targeted at the personal computer desktop user. With the growth of the Internet and information technology in general there has been a notable expansion in geospatial software options for the desktop, web and mobile platforms. A substantial part of this growth has been in Free and Open Source software.

	Some barriers to software adoption in general, especially in regards to web services is the amount technical knowledge and time required to assess, install and configure. In particular the time to learn how to setup and configure different applications prevents users from comparing all of their options before committing to a product. For this and other reasons only 1 or 2 GIS and remote sensing software packages are generally taught to students at many universities. This does a disservice to students by not adequately preparing them for jobs and research in geospatial Techonologies as outlined by the U.S. Department of Labor's Geospatial Competency Model \parencite{DepartmentofLabor2011}.
	
	Without getting into arguments about which software should be the one taught, a more reasonable approach seems to be to teach using several software packages. The differences between the options and the selection of tools more appropriate to tasks seems more in line with getting students to understand the implementations of geospatial concepts in software \parencite{Cammack2005}. It should also provide students with more options when making software decisions in their careers and not be limited by the idea that there is one standard GIS software that does everything. The biggest problems with this solution revolve around licensing and IT support time to manage a computer lab with complex software that most IT staff are unfamiliar with. 
	
	%From an academic perspective solving this issue would also open up possibilities for expanding instruction in geospatial technologies. Being able to provide a variety of software options ready for use in computer laboratories and for student use at home...
	The OSGeo Live project aims to alleviate some of these issues by providing a consistent platform on which to build educational curriculum that covers a wide variety of geospatial Technologies while reducing IT barriers and providing a resource which students can continue to use after they graduate. This project takes advantage of the ability to freely distribute free and open source software to ensure that anyone can download and use the software, eliminating licensing issues. It also builds upon years of experience in the Linux community creating bootable media \parencite{Negus2006,Knopper2006} which can run a whole computer and transitions that concept to also work in the new world of virtual machines both of which should decrease IT staff requirements.

%The initial goal of OSGeo Live was to make it easier for anyone to try geospatial tools, particularly Free and Open Source ones, by eliminating the need to install.
%I've been involved for over a year in developing a GIS workstation available in a variety of formats:
%a Live DVD, Live USB, Virtual Machines and live through the web. The workstation consists entirely of Free and Open Source Software (Over 40 applications), is available in multiple languages, contains sample data and only requires modest hardware to run. We've released 3 major revisions and are on schedule to create a new release every 6 months. Over 1000 copies have been handed out physically at the annual FOSS4G conference and just as many have been downloaded.
%
%There are lot of possible uses of OSGeo Live and in particular I want to focus on it's usefulness as a teaching tool. I've already piloted it's usage for a 2 hour lab as part of the International Seminar on Climate Change and Natural Resources Management through the US Forest Service and facilitated by ICE. It's also becoming the norm for workshops at FOSS4G. However I have yet to collect any formal feedback(informally: interest from developers wanting to add their software and volunteers to help translate), or thoroughly analyze the downloads/conference audience.


\section{Methods}
\subsection{What is OSGeo Live}
In collaboration with a few other members of the Open Source Geospatial Foundation(OSGeo), we created OSGeo Live (\url{http://live.osgeo.org}). It's an entire operating system pre-installed and configured with a suite of geospatial software applications. It can be run on most Intel based computers from either a DVD or specially formatted usb stick with no need to install anything \footnote{Some Exceptions apply to certain models of Mac computers}. Alternatively by installing virtual machine software \footnote{Only recommended for recent computers with adequate CPU and RAM to support, see \url{http://live.osgeo.org} for recommendations} and running a guest computer inside the the virtual machine application.

We have been building and releasing an updated version approximately every 6 months since September 2009. We initially started with approximately 3-5 people and now have over 20 active collaborators. Additionally more than 20 more people have begun to contribute translations of the documentation.  

Thanks to volunteers we also have the files hosted on 5 servers in California, Oregon, Germany and Greece which helps to decrease download times by using software to automatically redirect downloaders to the geographically closer servers.

\subparagraph{Contents:}
\begin{enumerate}
\item More than 40 geospatial applications that represent what the community considers some of the best Free and Open Source applications for desktop GIS, geographic web services, spatial databases, spatial statistics, web cartography, spatial software development libraries, and navigation tools.
\item Documentation is provided online and on the disc for each application in the form of a short description and a quick 5 minute tutorial on how to get started with a given application.
\item Several free data samples are provided from OpenStreetMap \footnote{\url{http://openstreetmap.org}}, NaturalEarthData \footnote{\url{http://NaturalEarthData.com}}, OSGeo/GRASS North Carolina Dataset
\end{enumerate}

\subparagraph{Internet Distribution Options:}
\begin{enumerate}
\item DVD ISO image file ready to burn or be booted in a Virtual Machine (Does not save between sessions, unless you write to another media)
\item DVD mini ISO optimized to be put onto a bootalbe usb flash drive. Instructions on how to create a bootable flash drive are provided on the website.
\item Virtual machine image compatible with common virtual machine software freely available. (e.g. VMWare player, VirtualBox, etc.)
\end{enumerate}

Additionally print runs of both DVDs and USB flash drives have been manufactured and given out a booths and in conference registration bags at several international geospatial conferences. Interest continues to grow in including OSGeo Live materials at more conferences.

\subsection{Use Cases}
Personally, I've used OSGeo Live twice to teach a 2 hour lesson on applications of GIS for studying climate change at an annual international workshop hosted by the Forest Service and the Information Center for the Environment. I've also successfully had OSGeo Live virtual machines installed in the UC Davis Informatics Center for optional use by several graduate level courses and used it as the platform for a conference workshop on QGIS features for Power Users.

I'm aware of other researchers and professionals also using OSGeo Live in similar ways. It has become the standard platform for workshops at the Free and Open Source for Geospatial (FOSS4G\footnote{\url{http://www.foss4g.org}}) conference.  Part of the final report on this project will include a collection of  case studies from other users to help create an understanding of how the technology could be used based on examples. 

\subsubsection{Geoweb}
I coordinated a course in the Spring of 2012, entitled GeoWeb. The intent of the course was to introduce students to the concept and implementations of Spatial Data Infrastructure (SDI). This was achieved through a series of lectures, lab exercises and a cumulative class wide group project to create a web map via several components of SDI. The lectures were structured to cover one topic per week, typically a single OGC standard (eg. WMS). After the lecture students would choose an application on OSGeo Live that implemented the standard/concept and in small groups work through the Quickstart provided by the OSGeo Live project. Students then filled out a review form each lesson about their experience with the software they had just explored. After introducing all of necessary topics (See table below), the class collectively decided on a theme for their web map, and divided themselves into small groups by concept/component so that each team would be responsible for one piece of SDI. These groups choose one option from the possible software to use.

The course was a group study with undergraduates and graduate students. Graduate students developed and provided one lecture each, and led each of the subgroups of the final project.

\begin{enumerate}
\item Web Services (WMS/WFS) - Geoserver
\item Tiling (OpenStreetmap) - Tilemill + TileStache
\item Database - Postgis
\item Web Client - OpenLayers
\end{enumerate}

\subparagraph{Materials}
Each student was provided with a Virtual Machine installed with OSGeo Live 6? via an existing GIS computer lab that had a typical Windows 7 host operating system. DVDs, USB sticks and instructions were provided for students wishing to use their own computers or do additional work at home.

\begin{enumerate}
\item WMS
\item WFS
\item WPS
\item Tiling
\item Web Clients
\item Spatial Databases
\end{enumerate}

\subparagraph{Survey Results}


\subsection{Log Analysis and Surveys}
Now that the OSGeo Live is well established and in use by the free and open source geospatial community I'd like to explore if it's had any success achieving it's stated goals. In addition to the educational opportunities participants in the project are also using OSGeo Live for marketing of their software \footnote{Marketing for market share not monetary gain since all of the software included is free.} and to quickly install fully functional GIS workstations or servers.

To resolve these questions I plan to first analyze the distribution of OSGeo Live via the Internet using the log files from the servers to geolocate all downloads to country, region and city (where possible) by Internet Protocol (IP) address. The analysis will also include an estimation of download time per user, and other descriptive statistics (e.g. number of downloads of each version of OSGeo Live over time). From this data I specifically want to analyze if the geographic distribution is limited primarily to North America and Europe or if up and coming nations who are just now beginning to invest heavily in geospatial are also showing interest (e.g. India, Brazil, China, etc.) as well as if it's even reaching smaller Southeast Asian countries or African nations.

%Then compare the speeds against the ookla netindex speeds and relative rankings for the similar time period (taking the average).\parencite{NetIndex2013}
 
Second I plan to create an online feedback survey to asses the community of users and provide a mechanism to improve the tools. Sampling will be done via 'Snowball' techniques using the  website, mailing lists and word of mouth at events (e.g. conference booths) to try and reach as many users as possible. The response rate itself should provide useful feedback on the interest and dispersion of the project. Hopefully the geographic distribution will be similar to the download logs and make it easier to eliminate false positives from the log analysis (e.g. web crawling bots in Russia)

\subparagraph{Outline of the Survey Focus:}
\begin{enumerate}
\item Better Understand the audience of product
	\begin{enumerate}
		\item What role they play in their 'company': implementer, decision maker, researcher, labor?
	\end{enumerate}
\item Who (Fields and Jobs) is using the product Where and for what purpose?
	\begin{enumerate}
		\item In Particular, Usage in learning Open Source GIS, GIS in general
	\end{enumerate}
\item Are we reaching people who are new to GIS?
	\begin{enumerate}
		\item Prior experience with GIS
		\item Prior experience with Open Source
		\item Prior experience with Open Source GIS
	\end{enumerate}
\item Does localization of material matter (Translation)?	
	\begin{enumerate}
		\item Location, Native location if different
		\item Languages, Primary Language if not English
		\item Does providing the documentation and software in a native language make it easier to use?
	\end{enumerate}
\end{enumerate}
		
%Analyze who has participated in the development of the product. Same survey or interviews?	
	
%Methods:
%\begin{enumerate}
%\item Analyze the download data to better understand who it's reaching (By at least Country level, City/Region where possible) and also obtain similar demographic data from conference attendance records where OSGeo Live is distributed.
%\item Survey (Opt-in, could require IRB short form for anonymous participation) of users of the disc \& web visitors
			
%Additional Background Notes: Adoption of Open Source solutions is occurring in nations like India, Spain, Brazil etc.. but is skipping the prior method of buying into the Commercial Product and jumping directly to use of Free and Open Source software including the funding of customization of existing applications and development of new applications.

%Note: Funding of the development of the solutions early on to ensure they meet the needs.

\printbibliography[heading=bibliography]

\end{document}